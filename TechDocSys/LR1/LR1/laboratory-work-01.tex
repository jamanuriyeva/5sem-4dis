\documentclass[14pt]{extreport}
\usepackage{fontspec}
\usepackage{polyglossia}
\setmainlanguage{russian}
\setotherlanguages{english}

% Установка шрифтов с поддержкой кириллицы
\setmainfont{CMU Serif} % Computer Modern Unicode с кириллицей
\newfontfamily\cyrillicfont{CMU Serif}[Script=Cyrillic]
\newfontfamily\cyrillicfonttt{CMU Typewriter Text}[Script=Cyrillic]
\newfontfamily\cyrillicfontsf{CMU Sans Serif}[Script=Cyrillic]

% Пакеты для выполнения заданий
\usepackage{geometry}
\usepackage{graphicx}
\usepackage{amsmath}
\usepackage{hyperref}
\usepackage{enumitem}
\usepackage{float}
\usepackage{fancyhdr}
\usepackage{tocloft}

% Настройка полей по ГОСТ (Задание 7)
\geometry{a4paper,left=30mm,right=10mm,top=20mm,bottom=20mm}

% Настройка отступов (Задание 7)
\setlength{\parindent}{1.25cm}
\usepackage{indentfirst}
\setlength{\parskip}{0pt}
\linespread{1.5}

% Настройка нумерации страниц (Задание 7)
\pagestyle{fancy}
\fancyhf{}
\fancyfoot[C]{\thepage}
\renewcommand{\headrulewidth}{0pt}
\fancypagestyle{plain}{\fancyhf{}\fancyfoot[C]{\thepage}}

% Настройка оглавления
\renewcommand{\cfttoctitlefont}{\hfill\bfseries}
\renewcommand{\cftaftertoctitle}{\hfill}

\title{Реферат на тему: \\[0.5cm] \textbf{Кибербезопасность: современные угрозы и методы защиты}}
\author{Студент: Нуриева Джамиля Эльхановна \\ Группа: 1}
\date{13.10.2025}

\begin{document}

% Титульная страница
\begin{titlepage}
    \centering
    {\large Реферат \par}
    \vspace{1cm}
    {\large по дисциплине: «Информационные технологии» \par}
    \vspace{1cm}
    {\Large \textbf{Кибербезопасность: современные угрозы и методы защиты} \par}
    \vfill
    \begin{flushright}
        \textbf{Выполнил:} \\
        Студент группы 1 \\
        Нуриева Д. Э. \\
        \vspace{1cm}
        \textbf{Проверил:} \\
        доц. Петров В. Г.
    \end{flushright}
    \vfill
    {\large Москва, 2025 г. \par}
\end{titlepage}

\tableofcontents

\chapter{Введение}
\label{chap:intro}

В современном мире, пронизанном цифровыми технологиями, информационные системы стали неотъемлемой частью практически всех сфер человеческой деятельности: от государственного управления и финансов до здравоохранения и повседневного общения. Рост объема циркулирующих данных, их концентрация и ценность закономерно привели к увеличению числа и сложности кибератак. \textbf{Кибербезопасность} перестала быть узкоспециализированной задачей IT-отделов и превратилась в стратегический приоритет для государств, корпораций и частных лиц.

\textbf{Актуальность} темы обусловлена постоянной эволюцией киберугроз, которые наносят значительный финансовый, репутационный и даже физический ущерб. Ежегодно фиксируются миллионы инцидентов, связанных с утечкой данных, выводом из строя критической инфраструктуры и цифровым мошенничеством.

\textbf{Целью} данного реферата является анализ современных угроз в области кибербезопасности и изучение методов защиты от них.

\textbf{Задачи} работы:
\begin{itemize}
    \item Классифицировать и охарактеризовать основные виды киберугроз.
    \item Рассмотреть существующие методы и технологии защиты информации.
    \item Выявить значение человеческого фактора в построении эффективной системы безопасности.
\end{itemize}

\chapter{Основная часть - 1}
\label{chap:main1}

\section{Раздел 1.1: Классификация современных киберугроз}
\label{sec:threats}

Современный ландшафт киберугроз отличается разнообразием и изощренностью. Условно их можно разделить на несколько категорий. \textit{Важно отметить, что многие современные атаки используют комбинацию различных методов для достижения своих целей}.

\subsection{Подраздел 1.1.1: Программные атаки}
\label{subsec:malware}

К данной категории относятся угрозы, реализуемые с помощью вредоносного программного обеспечения. \underline{Эти атаки представляют наибольшую опасность для рядовых пользователей и организаций}.

\begin{itemize}
    \item \textbf{Вредоносное ПО (Malware):} Общий термин для вирусов, червей, троянов, шпионского ПО и ransomware. Особую опасность представляют \textbf{шифровальщики (Ransomware)}, которые блокируют доступ к данным с требованием выкупа.
    \item \textbf{Фишинг и целевой фишинг (Spear Phishing):} Рассылка поддельных писем или сообщений, маскирующихся под легитимные источники, с целью кражи учетных данных или установки вредоносного ПО.
\end{itemize}

\subsection{Подраздел 1.1.2: Атаки на сетевую инфраструктуру}
\label{subsec:network}

Эти атаки нацелены на нарушение доступности и целостности сетевых ресурсов. \textbf{DDoS-атаки} могут парализовать работу даже крупных корпораций и государственных учреждений.

\begin{itemize}
    \item \textbf{DDoS-атаки (Distributed Denial of Service):} Направлены на перегрузку серверов или сетевых каналов огромным количеством запросов, что приводит к отказу в обслуживании легитимных пользователей.
    \item \textbf{Атаки "человек посередине" (Man-in-the-Middle):} Перехват и возможное изменение данных, передаваемых между двумя сторонами, оставшимися в неведении о вмешательстве.
\end{itemize}

\subsubsection{Подподраздел 1.1.2.1: Угрозы, связанные с IoT}
\label{subsubsec:iot}

Миллионы неправильно сконфигурированных "умных" устройств (камеры, датчики, бытовая техника) становятся легкой добычей для хакеров и могут быть использованы для создания бот-сетей (например, ботнета Mirai) для проведения масштабных DDoS-атак.

% Вставка изображения (Задание 8) - используем заглушку
\begin{figure}[H]
\centering
\includegraphics[width=0.8\textwidth]{example-image} 
\caption{Схема распространения современных киберугроз}
\label{fig:threats}
\end{figure}

На Рисунке \ref{fig:threats} показана типичная схема распространения киберугроз в современной цифровой среде. Как видно из диаграммы, атаки могут исходить из различных источников и использовать multiple векторы атаки.

\section{Раздел 2: Основные методы и средства защиты информации}
\label{sec:protection}

Для противодействия перечисленным угрозам используется комплексный подход, включающий технические и организационные меры. Эффективность различных методов защиты можно оценить по данным, представленным в Таблице \ref{tab:protection}.

% Вставка таблицы (Задание 9)
\begin{table}[H]
\centering
\begin{tabular}{|p{0.25\textwidth}|p{0.35\textwidth}|p{0.3\textwidth}|}
\hline
\textbf{Тип защиты} & \textbf{Описание} & \textbf{Эффективность} \\
\hline
\textbf{Антивирусное ПО} & Обнаружение и блокировка известных угроз & Высокая против известных угроз \\
\hline
\textbf{Межсетевые экраны} & Контроль сетевого трафика & Средняя-высокая \\
\hline
\textbf{Шифрование данных} & Защита конфиденциальности информации & Очень высокая \\
\hline
\textbf{MFA} & Многофакторная аутентификация & Высокая против несанкционированного доступа \\
\hline
\end{tabular}
\caption{Сравнительная характеристика методов защиты информации}
\label{tab:protection}
\end{table}

\subsection{Технические средства защиты}
К техническим средствам относятся:
\begin{itemize}
    \item \textbf{Антивирусное ПО и межсетевые экраны (Firewalls):} Базовый уровень защиты.
    \item \textbf{Системы обнаружения и предотвращения вторжений (IDS/IPS):} Мониторят сетевую активность в реальном времени.
    \item \textbf{Шифрование данных:} Защищает данные как при передаче, так и при хранении.
    \item \textbf{Многофакторная аутентификация (MFA):} Значительно усложняет несанкционированный доступ.
\end{itemize}

\subsection{Организационные меры}
Важнейшими организационными мерами являются:
\begin{itemize}
    \item Регулярное обучение и повышение осведомленности сотрудников.
    \item Разработка и соблюдение политики безопасности.
    \item Регулярное резервное копирование критически важных данных.
\end{itemize}

\section{Математические основы защиты информации}
\label{sec:math}

% Вставка формул (Задание 10)
В основе многих методов защиты информации лежат математические алгоритмы. Рассмотрим основные понятия:

\begin{equation}
E = mc^2
\end{equation}

где $E$ - энергия, $m$ - масса, $c$ - скорость света.

Для систем защиты информации важное значение имеет вероятность успешного предотвращения атаки:

\begin{equation}
P_s = 1 - \prod_{i=1}^{n}(1 - p_i)
\label{eq:security}
\end{equation}

где $p_i$ - вероятность нейтрализации угрозы на $i$-м уровне защиты, $n$ - количество уровней защиты.

Также важным показателем является эффективность системы обнаружения вторжений:

\begin{align}
\text{Precision} &= \frac{TP}{TP + FP} \\
\text{Recall} &= \frac{TP}{TP + FN}
\label{eq:metrics}
\end{align}

где $TP$ - истинно положительные срабатывания, $FP$ - ложные срабатывания, $FN$ - пропущенные угрозы.

Для получения дополнительной информации о современных стандартах безопасности рекомендуется посетить \href{https://www.nist.gov/cyberframework}{официальный сайт NIST Cybersecurity Framework}.

\chapter{Заключение}
\label{chap:conclusion}

Проведенный анализ позволяет сделать вывод, что кибербезопасность является динамичной и сложной областью, в которой угрозы постоянно эволюционируют. Не существует единого решения, способного обеспечить абсолютную защиту.

Эффективная стратегия безопасности должна быть \textbf{комплексной} и \textbf{многоуровневой}. Она обязана сочетать в себе передовые технические решения, четкие организационные процедуры и постоянное обучение сотрудников для минимизации человеческого фактора.

Только такой сбалансированный подход позволяет организациям и частным лицам не только противостоять известным угрозам, но и быть готовыми к новым вызовам цифровой эпохи. Будущее кибербезопасности лежит в области проактивной защиты, искусственного интеллекта для анализа угроз и непрерывной адаптации к изменяющемуся ландшафту рисков.

% Список литературы (Задание 5)
\section*{Список литературы}
\begin{enumerate}
    \item \textbf{Скимер, Б.} Кибербезопасность и кибероружие. — М.: ДМК Пресс, 2019.
    \item \textbf{Anderson, R.} Security Engineering: A Guide to Building Dependable Distributed Systems. — Wiley, 2020.
    \item \textbf{Столлингс, В.} Криптография и защита сетей. — М.: Вильямс, 2018.
    \item ГОСТ Р ИСО/МЭК 27001-2022 "Информационная технология. Методы и средства обеспечения безопасности. Системы менеджмента информационной безопасности. Требования".
    \item \textbf{Петров, А. А.} Современные методы защиты информации в корпоративных сетях // Журнал информационной безопасности, 2023. — № 4. — С. 45-52.
\end{enumerate}

\end{document}