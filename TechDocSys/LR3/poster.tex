\documentclass[a0paper,portrait]{tikzposter}

% ---------- Поддержка кириллицы ----------
\usepackage{fontspec}
\usepackage{polyglossia}
\setmainlanguage{russian}
\setotherlanguage{english}
\setmainfont{FreeSerif}
\setsansfont{FreeSans}
\setmonofont{FreeMono}

% ---------- Пакеты ----------
\usepackage{amsmath,amssymb,graphicx,tikz}

% ---------- Цветовая палитра в розовых тонах ----------
\definecolor{mypink}{RGB}{255,182,193}     % светло-розовый
\definecolor{mypink2}{RGB}{255,105,180}    % яркий розовый
\definecolor{myviolet}{RGB}{219,112,147}   % розово-фиолетовый
\definecolor{mybg}{RGB}{255,240,245}       % нежный фон

% ---------- Оформление темы ----------
\usetheme{Simple}
\useblockstyle[roundedcorners=25, linewidth=2pt, titleinnersep=8pt, bodyinnersep=10pt]{Default}
\definecolorstyle{pinkstyle}{
  \definecolor{backgroundcolor}{named}{mybg}
  \definecolor{blocktitlebgcolor}{named}{mypink2}
  \definecolor{blocktitlefgcolor}{named}{white}
  \definecolor{blockbodybgcolor}{named}{white}
  \definecolor{blockbodyfgcolor}{named}{black}
  \definecolor{titlefgcolor}{named}{black}
  \definecolor{framecolor}{named}{mybg}
  \definecolor{titlebgcolor}{named}{mybg} % ← фон заголовка белый!
}{}
\usecolorstyle{pinkstyle}

% ---------- Заголовок ----------
\title{Создание фрактального изображения с использованием OnlineTools}
\author{Нуриева Джамиля Эльхановна}
\institute{РГПУ им. А. И. Герцена}
\titlegraphic{
\vspace{-2cm}
\centering
\includegraphics[width=10cm]{logo.png}\\[1cm]
}

\begin{document}
\maketitle

% ---------- Abstract ----------
\block{Аннотация}{
Работа посвящена созданию фрактального изображения «Дерево Пифагора» с использованием онлайн-сервиса \textbf{OnlineTools}. 
Исследованы свойства самоподобных структур, параметры фрактала (угол ветвления, глубина рекурсии и масштаб), 
а также создано изображение с заданными параметрами.
}

% ---------- Introduction ----------
\block{Введение}{
Фракталы — это самоподобные геометрические структуры, в которых части подобны целому. 
Термин «фрактал» был предложен Бенуа Мандельбротом в 1975 году. 
Фракталы применяются в компьютерной графике, моделировании природных объектов и анализе сложных систем. 
В данной работе рассматривается построение фрактала «Дерево Пифагора».
}

% ---------- Methods ----------
\block{Методы}{
Для генерации фрактала использован веб-сервис \textbf{OnlineTools} 
(\texttt{https://onlinetools.com/math/generate-pythagoras-tree}). 
Он позволяет изменять параметры:
\begin{itemize}
\item угол ветвления $\alpha$;
\item глубину рекурсии $n$;
\item коэффициент масштабирования $k$;
\item цветовую палитру и размер canvas.
\end{itemize}

\vspace{6pt}
\begin{center}
\begin{tikzpicture}[scale=0.9]
  \filldraw[fill=mypink!60!white,draw=myviolet,thick] (0,0) rectangle (2,2);
  \begin{scope}[shift={(0.5,2)}, rotate=55]
    \filldraw[fill=myviolet!50!white,draw=myviolet,thick] (0,0) rectangle (1.4,1.4);
  \end{scope}
  \begin{scope}[shift={(1.5,2)}, rotate=10]
    \filldraw[fill=mypink2!40!white,draw=myviolet,thick] (0,0) rectangle (1.3,1.3);
  \end{scope}
  \node at (1,-0.8) {Схема «Дерева Пифагора»};
\end{tikzpicture}
\end{center}
}

% ---------- Results ----------
\block{Результаты}{
После настройки параметров ($n=10$, $\alpha=45^\circ$, $k=0.7$) 
получено фрактальное изображение, демонстрирующее самоподобие.  
Сервис позволяет экспортировать изображение и изменять палитры в реальном времени.  
\begin{itemize}
\item Пример параметров: 600×600 px, 10 итераций, градиентная палитра.  
\item При $n>12$ мелкие детали становятся неразличимыми.
\end{itemize}
}

% ---------- Conclusions ----------
\block{Выводы}{
\begin{itemize}
\item Изучены теоретические основы фракталов и «Дерева Пифагора»;
\item Проанализирован веб-сервис OnlineTools;
\item Создано изображение с параметрами $n=10$, $\alpha=45^\circ$, $k=0.7$;
\item Подтверждено свойство самоподобия.
\end{itemize}
}

% ---------- References ----------
\block{Список литературы}{
1. Сергеев Л. \emph{Фракталы: что это такое и какие они бывают}. — Skillbox, 2025.\\
2. Башкиров С. \emph{Что такое фракталы: сложные формы из простых правил}. — РБК, 2025.\\
3. Трошина Г. В. \emph{Математические методы обработки данных в инженерной практике}. — НГТУ, 2023.\\
4. Иудин Д. И., Копосов Е. В. \emph{Фракталы: от простого к сложному}. — ННГАСУ, 2012.
}

\end{document}
